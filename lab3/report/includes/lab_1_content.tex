\begin{center}
  \textbf{Отчёт лабораторной работы №\envReportLabNumber}
\end{center}

\textbf{Тема}:
<<\envReportTitle>>

\textbf{Цель}: ...

% = = = = = = = = = = = = = = = =

Беру методичку БНТУ \cite{MethodBntu}.

\begin{center}
  \textbf{Условие}
\end{center}

Компания контролирует 4 фабрики, производительность которых на неделю (в тыс. изделий) задается вектором $\bar{a} = (a_1, a_2 , a_3 , a_4 )$.
Компания заключила договоры с пятью заказчиками, потребность которых еженедельно (в тыс.изделий) задается вектором $\bar{b} = (b_1 , b_2 , b_3 , b_4 , b_5)$.
Стоимость транспортировки 1 тысячи изделий j-му заказчику  с i-ой фабрики-изготовителя задается матрицей $C = ||C_{ij}||_{4 \cdot 5}$.

Требуется:

\begin{enumerate}
  \item[1)] составить математическую модель задачи;
  \item[2)] привести ее к стандартной задаче ( с балансом);
  \item[3)] построить начальный опорный план;
  \item[4)] решить задачу методом потенциалов;
  \item[5)] проанализировать результаты решения.
\end{enumerate}

\begin{center}
  \textbf{Вариант 5}
\end{center}

$\bar{a} = (20,13,6,9)$, $\bar{b} = (8,11,7,5,12)$, $C= \begin{pmatrix}
  5&2&1&5&3\\
  8&7&8&1&2\\
  6&5&4&3&2\\
  3&6&1&5&4
\end{pmatrix}$

\begin{center}
  \textbf{Решение}
\end{center}

\textbf{Для того чтобы решить задачу я посмотрел несколько видео на YouTube}:
\begin{itemize}
  \item Решение транспортной задачи методом северо-западного угла без фиктивного склада и без фиктивного магазина \cite{Youtube1};
  \item Решение транспортной задачи методом северо-западного угла с фиктивным магазином \cite{Youtube2};
  \item Решение транспортной задачи методом северо-западного угла с фиктивным складом \cite{Youtube3};
  \item Решение транспортной задачи методом потенциалов без фиктивного склада и без фиктивного магазина \cite{Youtube4};
  \item Решение транспортной задачи методом потенциалов с фиктивным магазином \cite{Youtube5}.
\end{itemize}

\newpage

Выпишем условие в удобную таблицу (см. таблицу~\ref{tab:1}).

\begin{table}[h!]
  \scriptsize

  \centering

  \caption{Таблица к задаче}
  \label{tab:1}

  \begin{tabular}{|c||c|c|c|c|c||c|} 
    \hline
    $c_{ij}$  &$b_1$ &$b_2$  &$b_3$  &$b_4$  &$b_5$  &Склад (запасы) \\ \hline
    \hline
    $a_1$             &5  &2  &1  &5  &3  &20 \\  \hline
    $a_2$             &8  &7  &8  &1  &2  &13 \\  \hline
    $a_3$             &6  &5  &3  &3  &2  &6  \\  \hline
    $a_4$             &3  &6  &1  &5  &4  &9  \\  \hline
    \hline
    Магазин (заявки)  &8  &11 &7  &5  &12 &x  \\  \hline
  \end{tabular}
\end{table}

Проверяем баланс заявок и запасов.

$S_{\text{запасов}} = 20+13+6+9=48$

$S_{\text{заявок}} = 8+11+7+5+12=43$

$S_{\text{запасов}} \neq S_{\text{заявок}}$,
$S_{\text{запасов}} > S_{\text{заявок}}$,
$S_{\text{запасов}} - S_{\text{заявок}} = 48 - 43 = 5$

Добавим фиктивную заявку на ($S_{\text{запасов}} - S_{\text{заявок}} = 48 - 43 = $) 5 единиц.
Для неё нарисуем еще один столбик.
Фиктивной заявки соответствует нулевой тариф.
Заполним таблицу (см. таблицу~\ref{tab:2}).

\begin{table}[h!]
  \scriptsize

  \centering

  \caption{Таблица к задаче}
  \label{tab:2}

  \begin{tabular}{|c||c|c|c|c|c|c||c|} 
    \hline
    $c_{ij}$  &$b_1$ &$b_2$  &$b_3$  &$b_4$  &$b_5$ &$b_6$  &Склад (запасы) \\ \hline
    \hline
    $a_1$             &5  &2  &1  &5  &3  &0  &20 \\  \hline
    $a_2$             &8  &7  &8  &1  &2  &0  &13 \\  \hline
    $a_3$             &6  &5  &3  &3  &2  &0  &6  \\  \hline
    $a_4$             &3  &6  &1  &5  &4  &0  &9  \\  \hline
    \hline
    Магазин (заявки)  &8  &11 &7  &5  &12 &5  &x  \\  \hline
  \end{tabular}
\end{table}

Искать будем опорное решение методом минимального элемента.

Используем метод минимального элемента, так как решение методом минимального элемента дает решение более близкое к оптимальному,
чем метод северо-западного угла.

\textbf{Начнем с тарифа 1}. Он у трех элементов.

Пусть $c_{13} = 7$, тогда $c_{23}=0$, $c_{33}=0$, $c_{43}=0$, так как заявки магазина удовлетворены ($7=7+0+0+0$).

Остался ещё тариф 1. Пусть $c_{24}=5$, тогда $c_{54}=0$, $c_{34}=0$, $c_{44}=0$, так как заявки магазина удовлетворены ($5=0+5+0+0$).

\textbf{Берём тариф 2}. Он у трех элементов.

Пусть $c_{12}=11$, тогда $c_{22}=0$, $c_{32}=0$, $c_{42}=0$, так как заявки магазина удовлетворены ($11=11+0+0+0$).

Остался ещё тариф 2. Пусть $c_{25}=8$, тогда $c_{26}=0$, так как склад пуст ($13=0+0+0+5+8+0$).

Остался ещё тариф 2. Пусть $c_{35}=4$, тогда $c_{15}=0$, $c_{45}=0$, так как заявки магазина удовлетворены ($12=0+8+4+0$).

\textbf{Берём тариф 3}. Он у одного элемента.

Пусть $c_{41}=8$, тогда $c_{11}=0$, $c_{21}=0$, $c_{31}=0$, так как заявки магазина удовлетворены ($8=0+0+0+8$).

\newpage

\textbf{Берём тариф 0}. Он у четырёх элементов.

Пусть $c_{46}=1$, чтобы склад был пуст ($9=8+0+0+0+0+1$).

Пусть $c_{36}=2$, чтобы склад был пуст ($6=0+0+0+0+4+2$).

Пусть $c_{16}=2$, чтобы склад был пуст ($9=8+0+0+0+0+1$).

Данные мы записывали в таблицу (см. таблицу~\ref{tab:3}).

\begin{table}[h!]
  \scriptsize

  \centering

  \caption{Таблица к задаче}
  \label{tab:3}

  \begin{tabular}{|c||c|c|c|c|c|c||c|} 
    \hline
    $c_{ij}$  &$b_1$ &$b_2$  &$b_3$  &$b_4$  &$b_5$ &$b_6$  &Склад (запасы) \\ \hline
    \hline
    $a_1$             &5/-  &2/11 &1/7  &5/-  &3/-  &0/2  &20   \\  \hline
    $a_2$             &8/-  &7/-  &8/-  &1/5  &2/8  &0/-  &13   \\  \hline
    $a_3$             &6/-  &5/-  &3/-  &3/-  &2/4  &0/2  &6    \\  \hline
    $a_4$             &3/8  &6/-  &1/-  &5/-  &4/-  &0/1  &9    \\  \hline
    \hline
    Магазин (заявки)  &8    &11   &7    &5    &12   &5    &x    \\  \hline
  \end{tabular}
\end{table}

Находим стоимость данного начального опорного решения. Для этого перемножаем для базисных клеток количество перевозок на тариф и складываем.

$S = 2 \cdot 11 + 1 \cdot 7 + 0 \cdot 2 + 1 \cdot 5 + 2 \cdot 8 + 2 \cdot 4 + 0 \cdot 2 + 3 \cdot 8 + 0 \cdot 1 =\\
= 22 + 7 + 5 + 16 + 8 + 0 + 24 + 0 = 82 \text{ (денежные единицы)}$

Проверим начальное опорное решение на выраженность. Решение является не выраженым, если ($N = m + n + 1$) количество заполненных клеток
равно сумме количества строк и столбцов минус один.

$N = 9$ - количество базисных клеток (количество заполненных клеток).

$m = 4$ - количество строк.

$n = 6$ - количество столбцов.

$$
\scriptsize
9 \equiv 4+6-1 - \text{ опорное решение не вырожденное}
\begin{cases}
  N = 9 \\
  m+n-1 = 4+6-1 = 9
\end{cases}
$$

Задача в которой 4 строчки и 6 столбцов имеет решение в том случае, если количество заполненных клеток равно 9.
В нашем случае их действительно 9, значит данные опорное решение не вырожденное.

Если решение не вырождено, значит мы не делаем дополнительных манипуляций.

Нам необходимо вычислить потенциалы пунктов отправления и назначения для всех базисных клеток.

$$
\scriptsize
u_i + v_i = c_{ij} 
\begin{cases}
  u_1 + v_2 = 2; & u_3 + v_5 = 2; \\
  u_1 + v_3 = 1; & u_3 + v_6 = 0; \\
  u_1 + v_6 = 0; & \\
  \\
  u_2 + v_4 = 1; & u_4 + v_1 = 3; \\
  u_2 + v_5 = 2; & u_4 + v_6 = 0. \\
\end{cases}
$$

$$
\scriptsize
\begin{cases}
  u_1 = 0 \\
  u_1 + v_2 = 2 & \implies 0 + v_2 = 2 \implies v_2 = 2; \\
  u_1 + v_3 = 1 & \implies 0 + v_3 = 1 \implies v_3 = 1; \\
  u_1 + v_6 = 0 & \implies 0 + v_6 = 0 \implies v_6 = 0; \\
  u_3 + v_6 = 0 & \implies u_3 + 0 = 0 \implies u_3 = 0; \\
  u_4 + v_6 = 0 & \implies u_4 + 0 = 0 \implies u_4 = 0; \\
  u_4 + v_1 = 3 & \implies 0 + v_1 = 3 \implies v_1 = 3; \\
  u_3 + v_5 = 2 & \implies 0 + v_5 = 2 \implies v_5 = 2; \\
  u_2 + v_5 = 2 & \implies u_2 + 2 = 2 \implies u_2 = 0; \\
  u_2 + v_4 = 1 & \implies 0 + v_4 = 1 \implies v_4 = 1.
\end{cases}
$$

$$
\scriptsize
\begin{cases}
  u_1=0;  & v_1=3;\\
  u_2=0;  & v_2=2;\\
  u_3=0;  & v_3=1;\\
  u_4=0;  & v_4=1;\\
          & v_5=2;\\
          & v_6=0.
\end{cases}
$$

Вычислим оценки для всех не базисных клеток для того, чтобы определить является ли данный план перевозок оптимальным.

Если план перевозок больше, либо равно нулю, то план  перевозок оптимальный.

Переходим к вычислению оценок

$$
\scriptsize
\delta_{ij} = c_{ij} - (u_i + v_j)
\begin{cases}
  \delta_{11} = c_{11} - (u_1 + v_1) & \delta_{11} = 5 - (0 + 3) = 2 \geq 0 \\
  \delta_{14} = c_{14} - (u_1 + v_4) & \delta_{14} = 5 - (0 + 1) = 4 \geq 0 \\
  \delta_{15} = c_{15} - (u_1 + v_5) & \delta_{15} = 3 - (0 + 2) = 1 \geq 0 \\
  \delta_{21} = c_{21} - (u_2 + v_1) & \delta_{21} = 8 - (0 + 3) = 5 \geq 0 \\
  \delta_{22} = c_{22} - (u_2 + v_2) & \delta_{22} = 7 - (0 + 2) = 5 \geq 0 \\
  \delta_{23} = c_{23} - (u_2 + v_3) & \delta_{23} = 8 - (0 + 1) = 7 \geq 0 \\
  \delta_{26} = c_{26} - (u_2 + v_6) & \delta_{26} = 0 - (0 + 0) = 0 \geq 0 \\
  \delta_{31} = c_{31} - (u_3 + v_1) & \delta_{31} = 6 - (0 + 3) = 3 \geq 0 \\
  \delta_{32} = c_{32} - (u_3 + v_2) & \delta_{32} = 5 - (0 + 2) = 3 \geq 0 \\
  \delta_{33} = c_{33} - (u_3 + v_3) & \delta_{33} = 4 - (0 + 1) = 3 \geq 0 \\
  \delta_{34} = c_{34} - (u_3 + v_4) & \delta_{34} = 3 - (0 + 1) = 2 \geq 0 \\
  \delta_{42} = c_{42} - (u_4 + v_2) & \delta_{42} = 6 - (0 + 2) = 4 \geq 0 \\
  \delta_{43} = c_{43} - (u_4 + v_3) & \delta_{43} = 1 - (0 + 1) = 0 \geq 0 \\
  \delta_{44} = c_{44} - (u_4 + v_4) & \delta_{44} = 5 - (0 + 1) = 4 \geq 0 \\
  \delta_{45} = c_{45} - (u_4 + v_5) & \delta_{45} = 4 - (0 + 2) = 2 \geq 0 \\
\end{cases}
$$

Если б оценки были положительными, значит задача решена и данный план перевозок является оптимальным.
В нашем случае клетки положительные, значит план оптимален и его не надо оптимизировать.

\textbf{Ответ}:
Опорный план
$X_\text{оптим}= \begin{pmatrix}
  0 &11 &7  &0  &0  &2\\
  0 &0  &0  &5  &8  &0\\
  0 &0  &0  &0  &4  &2\\
  8 &0  &0  &0  &0  &1
\end{pmatrix}$;
стоимость данного начального опорного решения $S = 82$ денежные единицы.

% ----------------------

% Выпишем условие в удобную таблицу (см. таблицу~\ref{tab:1}).

% \begin{table}[h!]
%   \scriptsize

%   \centering

%   \caption{Таблица к задаче}
%   \label{tab:1}

%   \begin{tabular}{|c||c|c|c|c|c||c|} 
%     \hline
%     $c_{ij}$  &$b_1$ &$b_2$  &$b_3$  &$b_4$  &$b_5$  &Склад (запасы) \\ \hline
%     \hline
%     $a_1$             &5  &2  &1  &5  &3  &20 \\  \hline
%     $a_2$             &8  &7  &8  &1  &2  &13 \\  \hline
%     $a_3$             &6  &5  &3  &3  &2  &6  \\  \hline
%     $a_4$             &3  &6  &1  &5  &4  &9  \\  \hline
%     Магазин (заявки)  &8  &11 &7  &5  &12 &x   \\ \hline
%   \end{tabular}
% \end{table}

% Проверяем баланс заявок и запасов.

% $S_{\text{запасов}} = 20+13+6+9=48$

% $S_{\text{заявок}} = 8+11+7+5+12=43$

% $S_{\text{запасов}} \neq S_{\text{заявок}}$,
% $S_{\text{запасов}} > S_{\text{заявок}}$

% $S_{\text{запасов}} - S_{\text{заявок}} = 48 - 43 = 5$

% Добавим фиктивную заявку на $S_{\text{запасов}} - S_{\text{заявок}} = 48 - 43 = 5$ единиц.
% Для неё нарисуем еще один столбик.
% Фиктивной заявки соответствует нулевой тариф.
% Заполним таблицу (см. таблицу~\ref{tab:2}).

% \begin{table}[h!]
%   \scriptsize

%   \centering

%   \caption{Таблица к задаче}
%   \label{tab:2}

%   \begin{tabular}{|c||c|c|c|c|c|c||c|} 
%     \hline
%     $c_{ij}$  &$b_1$ &$b_2$  &$b_3$  &$b_4$  &$b_5$ &$b_6$  &Склад (запасы) \\ \hline
%     \hline
%     $a_1$             &5  &2  &1  &5  &3  &0  &20 \\  \hline
%     $a_2$             &8  &7  &8  &1  &2  &0  &13 \\  \hline
%     $a_3$             &6  &5  &3  &3  &2  &0  &6  \\  \hline
%     $a_4$             &3  &6  &1  &5  &4  &0  &9  \\  \hline
%     Магазин (заявки)  &8  &11 &7  &5  &12 &5  &x  \\ \hline
%   \end{tabular}
% \end{table}

% Искать будет опорное решение методом минимального элемента.

% Используем методом минимального элемента, так как решение методом минимального элемента дает решение более близкое к оптимальному,
% чем метод северо-западного угла.

% Начнем с тарифа 1. Он у трех элементов. Пусть $c_{43} = 7$, тогда $c_{13}=0$, $c_{23}=0$, $c_{33}=0$, так как заявки магазина удовлетворены ($7=0+0+0+7$).

% Остался ещё тариф 1. Пусть $c_{24}=5$, тогда $c_{14}=0$, $c_{34}=0$, $c_{44}=0$, так как заявки магазина удовлетворены ($5=0+5+0+0$).

% Берём тариф 2. Он у трех элементов. Пусть $c_{35}=6$, тогда $c_{31}=0$, $c_{32}=0$, $c_{33}=0$, $c_{34}=0$, $c_{36}=0$,
% так как на складе уже ничего нет ($6=0+0+0+0+6+0$).

% Остался ещё тариф 2. Пусть $c_{25}=6$, тогда $c_{15}=0$ и $c_{45}=0$, так как заявки магазина удовлетворены ($12=0+6+6+0$).

% Остался ещё тариф 2. Пусть $c_{12}=11$, тогда $c_{22}=0$, $c_{32}=0$, $c_{42}=0$, так как заявки магазина удовлетворены ($11=11+0+0+0$).

% Берём тариф 3. Пусть $c_{41}=2$.

% Берём тариф 5. Пусть $c_{11}=4$, чтобы хватило 2 единицы на $c_{21}$.

% Берём тариф 8. Пусть $c_{21}=2$, тогда $c_{26}=0$. Теперь склад пуст ($13=2+0+0+5+6+0$).

% Остался только фиктивный магазин с тарифом 0. Берём тариф 0. Пусть $c_{16}=5$, тогда $c_{46}=0$,
% так как заявки магазина удовлетворены ($5=5+0+0+0$).

% Данные мы записывали в таблицу (см. таблицу~\ref{tab:3}).

% \begin{table}[h!]
%   \scriptsize

%   \centering

%   \caption{Таблица к задаче}
%   \label{tab:3}

%   \begin{tabular}{|c||c|c|c|c|c|c||c|} 
%     \hline
%     $c_{ij}$  &$b_1$ &$b_2$  &$b_3$  &$b_4$  &$b_5$ &$b_6$  &Склад (запасы) \\ \hline
%     \hline
%     $a_1$             &5/4  &2/11 &1/-  &5/-  &3/-  &0/5  &20   \\  \hline
%     $a_2$             &8/2  &7/-  &8/-  &1/5  &2/6  &0/-  &13   \\  \hline
%     $a_3$             &6/-  &5/-  &3/-  &3/-  &2/6  &0/-  &6    \\  \hline
%     $a_4$             &3/2  &6/-  &1/7  &5/-  &4/-  &0/-  &9    \\  \hline
%     Магазин (заявки)  &8    &11   &7    &5    &12   &5    &x    \\ \hline
%   \end{tabular}
% \end{table}

% Находим стоимость данного начального опорного решения. Для этого перемножаем для базисных клеток количество перевозок на тариф и складываем.

% $S =5 \cdot 4 + 2 \cdot 11 + 0 \cdot 5 + 8 \cdot 2 + 1 \cdot 5 + 2 \cdot 6 + 2 \cdot 6 + 3 \cdot 2 + 1 \cdot 7 =\\
% = 20 + 22 + 0 + 16 + 5 + 12 + 12 + 6 + 7 = 100 \text{ (денежный единиц)}$

% Проверим начальное опорное решение на выраженность. Решение является не выраженым, если ($N = m + n + 1$) количество заполненных клеток
% равно сумме количества строк и столбцов минус один.

% $N = 9$ - количество базисных клеток (количество заполненных клеток).

% $m = 4$ - количество строк.

% $n = 6$ - количество столбцов.

% $$
% \scriptsize
% 9 \equiv 4+6-1 - \text{ опорное решение не вырожденное}
% \begin{cases}
%   N = 9 \\
%   m+n-1 = 4+6-1 = 9
% \end{cases}
% $$

% Задача в которой 4 строчки и 6 столбцов имеет решение в том случае, если количество заполненных клеток равно 9.
% В нашем случае их действительно 9, значит данные опорное решение не вырожденное.

% Если решение не вырождено, значит мы не делаем дополнительных манипуляций.

% Нам необходимо вычислить потенциалы пунктов отправления и назначения для всех базисных клеток.

% $$
% \scriptsize
% c_{ij} = u_i + v_i 
% \begin{cases}
%   u_1 + v_1 = 5; & u_3 + v_5 = 6; \\
%   u_1 + v_2 = 2; & u_4 + v_1 = 2; \\
%   u_1 + v_6 = 5; & u_4 + v_3 = 7; \\
%   u_2 + v_1 = 2; \\
%   u_2 + v_4 = 5; \\
%   u_2 + v_5 = 6. \\
% \end{cases}
% $$

% $u_1 = 0$.

% $u_1 + v_1 = 5$. $0 + v_1 = 5 \implies v_1 = 5$.

% $u_1 + v_2 = 2$. $0 + v_2 = 2 \implies v_2 = 2$.

% $u_1 + v_6 = 5$. $0 + v_6 = 5 \implies v_6 = 5$.

% $u_2 + v_1 = 5$. $u_2 + 5 = 2 \implies u_2 = -3$.

% $u_2 + v_4 = 5$. $-3 + v_4 = 5 \implies v_4 = 8$.

% $u_2 + v_5 = 6$. $-3 + v_5 = 6 \implies v_5 = 9$.

% $u_3 + v_5 = 6$. $u_3 + 9 = 6 \implies u_3 = -3$.

% $u_4 + v_1 = 2$. $u_4 + 5 = 2 \implies u_3 = -3$.

% $u_4 + v_3 = 7$. $-3 + v_3 = 7 \implies v_3 = 10$.

% $$
% \begin{cases}
%   u_1=0;  & v_1=5;\\
%   u_2=-3; & v_2=2;\\
%   u_3=-3; & v_3=10;\\
%   u_4=-3; & v_4=8;\\
%           & v_5=9;\\
%           & v_6=5.
% \end{cases}
% $$

% Вычислим оценки для всех не базисных клеток для того, чтобы определить является ли данный план перевозок оптимальным.

% Если план перевозок больше, либо равно нулю, то план  перевозок оптимальный.

% Переходим к вычислению оценок

% $$
% \scriptsize
% \delta_{ij} = c_{ij} - (u_i + v_j)
% \begin{cases}
%   \delta_{13} = c_{13} - (u_1 + v_3) & \delta_{13} = 1 - (0  + 10) = \underline{-9 \leq 0}  \\
%   \delta_{14} = c_{14} - (u_1 + v_4) & \delta_{14} = 5 - (0  +  8) = \underline{-3 \leq 0} \\
%   \delta_{15} = c_{15} - (u_1 + v_5) & \delta_{15} = 3 - (0  +  9) = \underline{-6 \leq 0} \\
%   \delta_{22} = c_{22} - (u_2 + v_2) & \delta_{22} = 7 - (-3 +  2) =  8 \\
%   \delta_{23} = c_{23} - (u_2 + v_3) & \delta_{23} = 8 - (-3 + 10) =  1 \\
%   \delta_{26} = c_{26} - (u_2 + v_6) & \delta_{26} = 0 - (-3 +  5) = \underline{-2 \leq 0} \\
%   \delta_{31} = c_{31} - (u_3 + v_1) & \delta_{31} = 6 - (-3 +  5) =  4 \\
%   \delta_{32} = c_{32} - (u_3 + v_2) & \delta_{32} = 5 - (-3 +  2) =  6 \\
%   \delta_{33} = c_{33} - (u_3 + v_3) & \delta_{33} = 4 - (-3 + 10) = \underline{-3 \leq 0} \\
%   \delta_{34} = c_{34} - (u_3 + v_4) & \delta_{34} = 3 - (-3 +  8) = \underline{-2 \leq 0} \\
%   \delta_{36} = c_{36} - (u_3 + v_6) & \delta_{36} = 0 - (-3 +  5) =  8 \\
%   \delta_{42} = c_{42} - (u_4 + v_2) & \delta_{42} = 6 - (-3 +  2) =  7 \\
%   \delta_{44} = c_{44} - (u_4 + v_4) & \delta_{44} = 5 - (-3 +  8) =  0 \\
%   \delta_{45} = c_{45} - (u_4 + v_5) & \delta_{45} = 4 - (-3 +  9) = \underline{-2 \leq 0} \\
%   \delta_{46} = c_{46} - (u_4 + v_6) & \delta_{46} = 0 - (-3 +  5) = \underline{-2 \leq 0} \\ 
% \end{cases}
% $$

% Если б оценки были положительными, значит задача решена и данный план перевозок является оптимальным.
% В нашем случае клетка 13, 14, 15, 26, 33, 34, 45, 46 имеет отрицательную оценку.
% Это говорит о том, что данный план перевозок не оптимальный и его можно улучшить.
% Для этого нужно вести клетку с отрицательной оценкой.
% Если таких клеток несколько, то вводится в базисный набор наибольшая по модулю отрицательная оценка.

% $max(|-9|, |-3|, |-6|, |-2|, |-3|, |-2|, |-2|, |-2|) = 9$ - клетка (1,3).

% План перевозок не оптимален. Введём в базисный набор клетку (1,3). Для клетки (1,3) необходимо построить цикл.
% Цикл будет следующим: (1,3) - (1,1) - (4,1) - (1,7) - (1,3). 

% Цикл является закрытым. То есть он начинается и заканчивается в одной клетке.

% Цикл начинается всегда в пустой не базисной клетке.

% Вершинами цикла или клетками, которыми он поворачивает могут быть только лишь заполненные базисные клетки.
% Цикл поворачивает только в базисных клетках.

% Цикл может перепрыгивать любое количество пустых, либо заполненных клеток.

% Цикл поворачивает под углом всегда 90 градусов.

% Для каждой клетки цикл можно провести единственным образом.

% В нашем случае простой цикл в виде прямоугольника (см. рисунок~\ref{fig:1}), но циклы могу быть пересекающиеся в виде восьмёрки,
% могут быть в виде сапога, может быть сапог с восьмёркой.

% Помечает вершины цикла по переменной плюс и минус начиная с пустой базисной клетки.
% Пустая базисная клетка - это клетка (1,3). Помечаю её плюсом. Дальше по переменно плюс и минус.

% Клетки помеченные плюсом образуют положительную полу-цепь транспортной задачи.

% Клетки помеченные минусом образуют отрицательную полу-цепь транспортной задачи.

% Нам надо среди отрицательной полу цепи найти наименьшее количество перевозок. Перевозки - это те числа, которые мы ставили.

% $min(4, 7) = 4$.

% \begin{figure}[!htb]
%   \centering

%   \includegraphics[width=16cm]
%   {inc/cycle1.png}

%   \caption{Таблица к задаче}
%   \label{fig:1}
% \end{figure}

% Записываем новую транспортную таблицу (см. таблицу~\ref{tab:4}) и проводим корректировку транспортной задачи.
% Тарифы записываем без изменений. Перевозки, которые не вошли в цикл, переписывает без изменений.
% Если цикл перепрыгивает, то цикл не как на неё не повлиял и её переписали без изменений.
% К элементам положительной полу-цепи прибавляем 4. От элементам отрицательной полу-цепи отнимает 4.
% При этом ячейка, которая раньше была пустая станет заполненная, а ячейка, в которой было наименьшее количество перевозок, станет пустой.

% \begin{table}[h!]
%   \scriptsize

%   \centering

%   \caption{Таблица к задаче}
%   \label{tab:4}

%   \begin{tabular}{|c||c|c|c|c|c|c||c|} 
%     \hline
%     $c_{ij}$  &$b_1$ &$b_2$  &$b_3$  &$b_4$  &$b_5$ &$b_6$  &Склад (запасы) \\ \hline
%     \hline
%     $a_1$             &5/-  &2/11 &1/4  &5/-  &3/-  &0/5  &20   \\  \hline
%     $a_2$             &8/2  &7/-  &8/-  &1/5  &2/6  &0/-  &13   \\  \hline
%     $a_3$             &6/-  &5/-  &3/-  &3/-  &2/6  &0/-  &6    \\  \hline
%     $a_4$             &3/6  &6/-  &1/3  &5/-  &4/-  &0/-  &9    \\  \hline
%     Магазин (заявки)  &8    &11   &7    &5    &12   &5    &x    \\ \hline
%   \end{tabular}
% \end{table}

% Посчитает какая новая стоимость перевозки с новым решением. Для этого перемножаем количество перевозок на тариф.

% $S = 2 \cdot 11 + 1 \cdot 4 + 0 \cdot 5 + 8 \cdot 2 + 1 \cdot 5 + 2 \cdot 6 + 2 \cdot 6 + 3 \cdot 6 + 1 \cdot 3 = \\
% =22+4+0+16+5+12+12+18+3=92 \text{ (денежных единиц)}
% $

% Проверим является ли решение оптимальным.
% Вычисляем потенциалы для базисных клеток.
% Вычисляем оценки для всех небазисных.
% Если оценки станут положительными, то это говорит о том, что план перевозок стал оптимальным.

% $$
% \scriptsize
% c_{ij} = u_i + v_i 
% \begin{cases}
%   u_1 + v_2 = 11; & u_3 + v_5 = 6; \\
%   u_1 + v_3 = 4;  & u_4 + v_1 = 6; \\
%   u_1 + v_6 = 5;  & u_4 + v_3 = 3; \\
%   u_2 + v_1 = 2; \\
%   u_2 + v_4 = 5; \\
%   u_2 + v_5 = 6. \\
% \end{cases}
% $$

% $u_1 = 0$.

% $u_1 + v_2 = 11$. $0 + v_2 = 11 \implies v_2 = 11$.

% $u_1 + v_3 = 4$. $0 + v_3 = 4 \implies v_3 = 4$.

% $u_1 + v_6 = 5$. $0 + v_6 = 5 \implies v_6 = 5$.

% $u_4 + v_3 = 3$. $u_4 + 4 = 3 \implies u_4 = -1$.

% $u_4 + v_1 = 6$. $-1 + v_1 = 6 \implies v_1 = 7$.

% $u_2 + v_1 = 2$. $u_2 + 7 = 2 \implies u_2 = -5$.

% $u_2 + v_4 = 5$. $-5 + v_4 = 5 \implies v_4 = 10$.

% $u_2 + v_5 = 6$. $-5 + v_5 = 6 \implies v_5 = 11$.

% $u_3 + v_5 = 6$. $u_3 + 11 = 6 \implies u_3 = -5$.


% $$
% \scriptsize
% \begin{cases}
%   u_1=0;  & v_1=7;\\
%   u_2=-5; & v_2=11;\\
%   u_3=-5; & v_3=4;\\
%   u_4=-1; & v_4=10;\\
%           & v_5=11;\\
%           & v_6=5.
% \end{cases}
% $$

% $$
% \scriptsize
% \delta_{ij} = c_{ij} - (u_i + v_j)
% \begin{cases}
%   \delta_{11} = c_{11} - (u_1 + v_1) & \delta_{11} = 5 - (0  +  7) = \underline{-2 \leq 0}  \\
%   \delta_{14} = c_{14} - (u_1 + v_4) & \delta_{14} = 5 - (0  + 10) = \underline{-5 \leq 0} \\
%   \delta_{15} = c_{15} - (u_1 + v_5) & \delta_{15} = 3 - (0  + 11) = \underline{-8 \leq 0} \\
%   \delta_{22} = c_{22} - (u_2 + v_2) & \delta_{22} = 7 - (-5 + 11) = 1 \\
%   \delta_{23} = c_{23} - (u_2 + v_3) & \delta_{23} = 8 - (-5 +  4) = 9 \\
%   \delta_{26} = c_{26} - (u_2 + v_6) & \delta_{26} = 0 - (-5 +  5) = 0 \\
%   \delta_{31} = c_{31} - (u_3 + v_1) & \delta_{31} = 6 - (-5 +  7) = 4 \\
%   \delta_{32} = c_{32} - (u_3 + v_2) & \delta_{32} = 5 - (-5 + 11) = \underline{-1 \leq 0} \\
%   \delta_{33} = c_{33} - (u_3 + v_3) & \delta_{33} = 4 - (-5 +  4) = 5 \\
%   \delta_{34} = c_{34} - (u_3 + v_4) & \delta_{34} = 3 - (-5 + 10) = \underline{-2 \leq 0} \\
%   \delta_{36} = c_{36} - (u_3 + v_6) & \delta_{36} = 0 - (-5 +  5) = 0 \\
%   \delta_{42} = c_{42} - (u_4 + v_2) & \delta_{42} = 6 - (-1 + 11) = \underline{-4 \leq 0} \\
%   \delta_{44} = c_{44} - (u_4 + v_4) & \delta_{44} = 5 - (-1 + 10) = \underline{-4 \leq 0} \\
%   \delta_{45} = c_{45} - (u_4 + v_5) & \delta_{45} = 4 - (-1 + 11) = \underline{-6 \leq 0} \\
%   \delta_{46} = c_{46} - (u_4 + v_6) & \delta_{46} = 0 - (-1 +  5) = \underline{-4 \leq 0} \\ 
% \end{cases}
% $$

% $max(|-2|, |-5|, |-8|, |-1|, |-2|, |-4|, |-4|, |-6|, |-4|) = 8$

% План перевозок не оптимален. Введём в базисный набор клетку (1,5). Для клетки (1,5)
% необходимо построить цикл. Цикл будет следующим: (1,5) - (2,5) - (2,1) - (4,1) - (3,3) - (1,3) - (1,5).

% В нашем случае цикл восьмёркой (см. рисунок~\ref{fig:2}).

% \begin{figure}[!htb]
%   \centering

%   \includegraphics[width=16cm]
%   {inc/cycle2.png}

%   \caption{Таблица к задаче}
%   \label{fig:2}
% \end{figure}

% Нам надо среди отрицательной полу-цепи найти наименьшее количество перевозок. 

% $min(6, 6, 4) = 4$

% Записываем новую транспортную таблицу (см. таблицу~\ref{tab:5}) и проводим корректировку транспортной задачи.
% Тарифы записываем без изменений. Перевозки, которые не вошли в цикл, переписывает без изменений.
% Если цикл перепрыгивает, то цикл не как на неё не повлиял и её переписали без изменений.
% К элементам положительной полу-цепи прибавляем 4. От элементам отрицательной полу-цепи отнимает 4.
% При этом ячейка, которая раньше была пустая станет заполненная, а ячейка, в которой было наименьшее количество перевозок, станет пустой.

% \begin{table}[h!]
%   \scriptsize

%   \centering

%   \caption{Таблица к задаче}
%   \label{tab:5}

%   \begin{tabular}{|c||c|c|c|c|c|c||c|} 
%     \hline
%     $c_{ij}$  &$b_1$ &$b_2$  &$b_3$  &$b_4$  &$b_5$ &$b_6$  &Склад (запасы) \\ \hline
%     \hline
%     $a_1$             &5/-  &2/11 &1/-  &5/-  &3/4  &0/5  &20   \\  \hline
%     $a_2$             &8/6  &7/-  &8/-  &1/5  &2/2  &0/-  &13   \\  \hline
%     $a_3$             &6/-  &5/-  &3/-  &3/-  &2/6  &0/-  &6    \\  \hline
%     $a_4$             &3/2  &6/-  &1/7  &5/-  &4/-  &0/-  &9    \\  \hline
%     Магазин (заявки)  &8    &11   &7    &5    &12   &5    &x    \\ \hline
%   \end{tabular}
% \end{table}


% Посчитает какая новая стоимость перевозки с новым решением. Для этого перемножаем количество перевозок на тариф.

% $S = 2 \cdot 11 + 3 \cdot 4 + 0 \cdot 5 + 8 \cdot 6 + 1 \cdot 5 + 2 \cdot 2 + 2 \cdot 6 + 3 \cdot 2 + 1 \cdot 7 = \\
% =22+12+0+48+5+4+12+6+7=116 \text{ (денежных единиц)}
% $

% Проверим является ли решение оптимальным.
% Вычисляем потенциалы для базисных клеток.
% Вычисляем оценки для всех небазисных.
% Если оценки станут положительными, то это говорит о том, что план перевозок стал оптимальным.

% $$
% \scriptsize
% c_{ij} = u_i + v_i 
% \begin{cases}
%   u_1 + v_2 = 11; & u_3 + v_5 = 6; \\
%   u_1 + v_3 = 4;  & u_4 + v_1 = 6; \\
%   u_1 + v_6 = 5;  & u_4 + v_3 = 3; \\
%   u_2 + v_1 = 2; \\
%   u_2 + v_4 = 5; \\
%   u_2 + v_5 = 6. \\
% \end{cases}
% $$
