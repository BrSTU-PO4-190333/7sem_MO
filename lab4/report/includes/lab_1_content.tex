\begin{center}
  \textbf{Отчёт лабораторной работы №\envReportLabNumber}
\end{center}

\textbf{Тема}:
<<\envReportTitle>>

\textbf{Цель}: ...

% = = = = = = = = = = = = = = = =

\begin{center}
  \textbf{Условие}
\end{center}

На сети, изображенной ниже (см. рисунок~\ref{fig:task}), сформировать поток максимальной мощности,
направленный из истока I в сток S при условии,
что пропускные способности всех ребер в обоих направлениях одинаковы.
Выписать ребра, образующие на сети разрез минимальной пропускной способности.
Ниже приводятся вариант пропускных способностей ребер (см. таблицу~\ref{tab:var5}).

\begin{figure}[!h]
  \centering

  \includegraphics[height=4cm]
  {assets/export/MO_lab4_graph-Page-1.pdf}

  \caption{Граф к заданию}

  \label{fig:task}
\end{figure}

\begin{table}[h!]
  \scriptsize

  \centering

  \caption{Таблица к задаче}
  \label{tab:var5}

  \begin{tabular}{|c|c|c|c|c|c|c|c|c|} 
    \hline
    Вариант &$r_{13}$ &$r_{12}$ &$r_{23}$ &$r_{34}$ &$r_{25}$ &$r_{45}$ &$r_{46}$ &$r_{56}$ \\ \hline
    5       &3        &7        &6        &4        &3        &7        &5        &4        \\ \hline
  \end{tabular}
\end{table}

\begin{center}
  \textbf{Решение}
\end{center}

Решать буду как в методичке БрГТУ \cite{MethodBrstu}.

Для начала выпишу на граф (см. рисунок~\ref{fig:task}) данные моего варианта (см. таблицу~\ref{tab:var5}) - и
получу новый граф (см. рисунок~\ref{fig:task_var5}).

\begin{figure}[!h]
  \centering

  \includegraphics[height=5cm]
  {assets/export/MO_lab4_graph_var5-Page-1.pdf}

  \caption{Граф для варианта 5}

  \label{fig:task_var5}
\end{figure}

\newpage

1. Строим по графу таблицу R (см. таблицу~\ref{tab:Rtemp}).
\begin{table}[h!]
  \scriptsize

  \centering

  \caption{$(R)$}

  \label{tab:Rtemp}

  \begin{tabular}{|c|cccccc|} 
    \hline
    R&1&2&3&4&5&6 \\ \hline
    1&0&7&3&0&0&0 \\ 
    2&7&0&6&0&3&0 \\ 
    3&3&6&0&4&0&0 \\ 
    4&0&0&4&0&7&5 \\ 
    5&0&3&0&7&0&4 \\ 
    6&0&0&0&5&4&0 \\ \hline
  \end{tabular}

\end{table}

2. Выбираю пути 1-3-4-6 и 1-2-5-6.
Результат смотри на рисунке~\ref{fig:point2}.

\begin{figure}[!h]
  \centering

  \includegraphics[height=4cm]
  {assets/export/MO_lab4_graph_var5_2-Page-1.pdf}

  \caption{Выбираю пути}

  \label{fig:point2}
\end{figure}

3. Высчитываю минимальный поток.

$$
\begin{cases}
  (I-1-3-4-6-S): &min(r_{13}; r_{34}; r_{46}) = min(3;4;5)=3 \\
  (I-1-2-5-6-S): &min(r_{12}; r_{25}; r_{56}) = min(7;3;4)=3 \\
\end{cases}
$$

4. Минимальный поток показываю на графе. Результат смотри на рисунке~\ref{fig:point4}.

\begin{figure}[!h]
  \centering

  \includegraphics[height=4cm]
  {assets/export/MO_lab4_graph_var5_3-Page-1.pdf}

  \caption{Пометили потоки минимальным}

  \label{fig:point4}
\end{figure}

5. Считаю f по графу с минимальными потоками (см. рисунок~\ref{fig:point4}).

$$
f = \sum _j X_{Ij} = \sum _i X_{iS}
$$

$$
6 \equiv 6
\begin{cases}
  f(X^0) = x_{13}^0 + x_{12}^0=3+3=6 \\
  f(X^0) = x_{46}^0 + x_{56}^0=3+3=6 \\
\end{cases}
$$

\newpage

6. Строю таблицу $X^0$ (см. таблицу~\ref{tab:X0temp}) по графу с минимальными потоками (см. рисунок~\ref{fig:point4}).

\begin{table}[h!]
  \scriptsize

  \centering

  \caption{$(X^0)$}

  \label{tab:X0temp}

  \begin{tabular}{|c|cccccc|} 
    \hline
    $X^0$&1&2&3&4&5&6 \\ \hline
    1&0 & 3& 3& 0& 0&0 \\ 
    2&-3& 0& 0& 0& 3&0 \\ 
    3&-3& 0& 0& 3& 0&0 \\ 
    4&0 & 0&-3& 0& 0&3 \\ 
    5&0 &-3& 0& 0& 0&3 \\ 
    6&0 & 0& 0&-3&-3&0 \\ \hline
  \end{tabular}

\end{table}

7. Отнимаю от таблицы $R$ (см. таблицу~\ref{tab:R}) таблицу $X^0$ (см. таблицу~\ref{tab:X0}) - и получаем таблицу $R-X^0$ (см. таблицу~\ref{tab:RX0}).

\begin{table}[h!]
  \scriptsize

  \centering
  
  \begin{minipage}{0.32\textwidth}
    \centering

    \caption{$(R)$}

    \label{tab:R}

    \begin{tabular}{|c|cccccc|} 
      \hline
      R&1&2&3&4&5&6 \\ \hline
      1&0&7&3&0&0&0 \\ 
      2&7&0&6&0&3&0 \\ 
      3&3&6&0&4&0&0 \\ 
      4&0&0&4&0&7&5 \\ 
      5&0&3&0&7&0&4 \\ 
      6&0&0&0&5&4&0 \\ \hline
    \end{tabular}
  \end{minipage}
  \begin{minipage}{0.32\textwidth}
    \centering

    \caption{$(X^0)$}

    \label{tab:X0}

    \begin{tabular}{|c|cccccc|} 
      \hline
      $X^0$&1&2&3&4&5&6 \\ \hline
      1&0 & 3& 3& 0& 0&0 \\ 
      2&-3& 0& 0& 0& 3&0 \\ 
      3&-3& 0& 0& 3& 0&0 \\ 
      4&0 & 0&-3& 0& 0&3 \\ 
      5&0 &-3& 0& 0& 0&3 \\ 
      6&0 & 0& 0&-3&-3&0 \\ \hline
    \end{tabular}
  \end{minipage}
  \begin{minipage}{0.32\textwidth}
    \centering

    \caption{$(R - X^0)$}

    \label{tab:RX0}

    \begin{tabular}{|c|cccccc|} 
      \hline
      $R-X^0$&1&2&3&4&5&6 \\ \hline
      1&0 & 4& 0& 0& 0&0 \\ 
      2&10& 0& 6& 0& 0&0 \\ 
      3& 6& 6& 0& 1& 0&0 \\ 
      4&0 & 0& 7& 0& 7&2 \\ 
      5&0 & 6& 0& 7& 0&1 \\ 
      6&0 & 0& 0& 8& 7&0 \\ \hline
    \end{tabular}
  \end{minipage}

\end{table}

8. Строю граф по таблице $R-X^0$ (см. таблицу~\ref{tab:RX0}).
Результат смотри на рисунке~\ref{fig:point3}.

\begin{figure}[!h]
  \centering

  \includegraphics[height=4cm]
  {assets/export/MO_lab4_graph_var5_4-Page-1.pdf}

  \caption{Граф по таблице $R-X^0$}

  \label{fig:point3}
\end{figure}

9. $\Delta = min_{1-2-3-4-6}=(4;6;1;2)=1$

10. $f(X^1) = f(X^0) + \Delta = 6+1=7$

11. К таблице $X^0$ (см. таблицу~\ref{tab:X0}) добавим $\Delta = 1$ к элементам $x_{12}^0$, $x_{23}^0$, $x_{34}^0$, $x_{46}^0$.

У таблицы $X^0$ (см. таблицу~\ref{tab:X0}) отнимем $-\Delta = -1$ у элементов $x_{21}^0$, $x_{32}^0$, $x_{43}^0$, $x_{64}^0$.

Результаты запишем в таблицу $X^1$ (см. таблицу~\ref{tab:X1temp}).

\begin{table}[h!]
  \scriptsize

  \centering

  \caption{$(X^1)$}

  \label{tab:X1temp}

  \begin{tabular}{|c|cccccc|} 
    \hline
    $X^1$&1&2&3&4&5&6 \\ \hline
    1&0 & 4& 3& 0& 0&0 \\ 
    2&-4& 0& 1& 0& 3&0 \\ 
    3&-3&-1& 0& 4& 0&0 \\ 
    4&0 & 0&-4& 0& 0&4 \\ 
    5&0 &-3& 0& 0& 0&3 \\ 
    6&0 & 0& 0&-4&-3&0 \\ \hline
  \end{tabular}

\end{table}

12. Отнимем от таблицы $R$ (см. таблицу~\ref{tab:R1}) таблицу $X^1$ (см. таблицу~\ref{tab:X1}) - и получим
таблицу $R-X^1$ (см. таблицу~\ref{tab:RX1}).

\begin{table}[h!]
  \scriptsize

  \centering
  
  \begin{minipage}{0.32\textwidth}
    \centering

    \caption{$(R)$}

    \label{tab:R1}

    \begin{tabular}{|c|cccccc|} 
      \hline
      R&1&2&3&4&5&6 \\ \hline
      1&0&7&3&0&0&0 \\ 
      2&7&0&6&0&3&0 \\ 
      3&3&6&0&4&0&0 \\ 
      4&0&0&4&0&7&5 \\ 
      5&0&3&0&7&0&4 \\ 
      6&0&0&0&5&4&0 \\ \hline
    \end{tabular}
  \end{minipage}
  \begin{minipage}{0.32\textwidth}
    \centering

    \caption{$(X^1)$}

    \label{tab:X1}

    \begin{tabular}{|c|cccccc|} 
      \hline
      $X^1$&1&2&3&4&5&6 \\ \hline
      1&0 & 4& 3& 0& 0&0 \\ 
      2&-4& 0& 1& 0& 3&0 \\ 
      3&-3&-1& 0& 4& 0&0 \\ 
      4&0 & 0&-4& 0& 0&4 \\ 
      5&0 &-3& 0& 0& 0&3 \\ 
      6&0 & 0& 0&-4&-3&0 \\ \hline
    \end{tabular}
  \end{minipage}
  \begin{minipage}{0.32\textwidth}
    \centering

    \caption{$(R - X^1)$}

    \label{tab:RX1}

    \begin{tabular}{|c|cccccc|} 
      \hline
      $R-X^1$&1&2&3&4&5&6 \\ \hline
      1&0 & 3& 0& 0& 0&0 \\ 
      2&11& 0& 5& 0& 0&0 \\ 
      3&6 & 7& 0& 0& 0&0 \\ 
      4&0 & 0& 8& 0& 7&1 \\ 
      5&0 & 6& 0& 7& 0&1 \\ 
      6&0 & 0& 0& 9& 7&0 \\ \hline
    \end{tabular}
  \end{minipage}
\end{table}

13. Строю граф по таблице $R-X^1$ (см. таблицу~\ref{tab:RX1}).
Результат смотри на рисунке~\ref{fig:point4}.

\begin{figure}[!h]
  \centering

  \includegraphics[height=3cm]
  {assets/export/MO_lab4_graph_var5_5-Page-1.pdf}

  \caption{Граф по таблице $R-X^1$}

  \label{fig:point4}
\end{figure}

14. Строю граф по таблице $X^1$ (см. таблицу~\ref{tab:RX1}) по элементам сверху главной диагонали и разделяю граф линией.
Результат смотри на рисунке~\ref{fig:point5}.

\begin{figure}[!h]
  \centering

  \includegraphics[height=3cm]
  {assets/export/MO_lab4_graph_var5_6-Page-1.pdf}

  \caption{Граф по таблице $X^1$}

  \label{fig:point5}
\end{figure}

15. Правильность построения максимального потока можно проверить, используя теорему Форда-Фалкерсона,
согласно которой на любой сети максимальная величина потока истока в сток равна минимальной пропускной способности
разреза на сети, отделяющий исток от стока.

Разрез смотри на рисунке~\ref{fig:point5}.

$$
7 \equiv 7
\begin{cases}
  f(X^1) = f(X^0) + \Delta = 6+1=7 \\
  X_{34}^1 + X_{25}^1 = 4+3 = 7 \\
\end{cases}
$$

\textbf{Ответ}:
$X^1= \begin{pmatrix}
  0 & 4& 3& 0& 0&0 \\ 
  -4& 0& 1& 0& 3&0 \\ 
  -3&-1& 0& 4& 0&0 \\ 
  0 & 0&-4& 0& 0&4 \\ 
  0 &-3& 0& 0& 0&3 \\ 
  0 & 0& 0&-4&-3&0 \\
\end{pmatrix}$,
$R - X^1= \begin{pmatrix}
  0 & 3& 0& 0& 0&0 \\ 
  11& 0& 5& 0& 0&0 \\ 
  6 & 7& 0& 0& 0&0 \\ 
  0 & 0& 8& 0& 7&1 \\ 
  0 & 6& 0& 7& 0&1 \\ 
  0 & 0& 0& 9& 7&0 \\
\end{pmatrix}$.
