\begin{center}
  \textbf{Отчёт лабораторной работы №\envReportLabNumber}
\end{center}

\textbf{Тема}:
<<\envReportTitle>>

\textbf{Цель}: ...

% = = = = = = = = = = = = = = = =

\begin{center}
  \textbf{Условие}
\end{center}

Для производства трех видов продукции используются три вида сырья.
Нормы затрат каждого из видов сырья на единицу продукции данного вида, запасы сырья,
а также прибыль с единицы продукции приведены в таблицах вариантов.
Определить план выпуска продукции для получения максимальной прибыли при заданном дополнительном ограничении.
Оценить каждый из видов сырья, используемых для производства продукции.

Требуется:

\begin{enumerate}
  \item[1)] построить математическую модель задачи;
  \item[2)] выбрать метод решения и привести задачу к канонической форме;
  \item[3)] решить задачу(симплекс-методом);
  \item[4)] проанализировать результаты решения;
  \item[5)] составить к данной задаче двойственную и, используя соответствие перемен­ных, выписать ответ двойственной задачи;
  \item[6)] дать экономическую интерпретацию двойственных оценок;
  \item[7)] указать наиболее дефицитный и избыточный ресурс, если он есть.
\end{enumerate}

\begin{center}
  \textbf{Вариант 5}
\end{center}

Данные смотри в таблице~\ref{tab:1}.

\begin{table}[h!]
  %\scriptsize

  \centering

  \caption{Таблица к задаче}
  \label{tab:1}

  \begin{tabular}{|c||c|c|c||c|} 
    \hline
    Сырье/Продукция &A  &B  &C  &Запасы сырья, ед. \\ \hline
    \hline
    I               &-  &1  &1  &8  \\  \hline
    II              &1  &1  &-  &5  \\  \hline
    III             &-  &2  &1  &12 \\  \hline
    \hline
    Прибыль, ден.ед.&1  &5  &2  &x  \\  \hline
  \end{tabular}
\end{table}

\begin{center}
  \textbf{Решение}
\end{center}

Решать буду симплекс таблице как показывали на лекции по дицсиплине <<Методы оптимизации>> в БрГТУ.

Пример, который показывали на лекции я нашел в методичке БНТУ \cite{MethodBntu}.

Отличие решения, что в методички от того, что было на лекции - это другая таблица.
В методичке БНТУ не удобная таблица.
Буду рисовать таблицу, как показали на лекции в БрГТУ.

\newpage

\textbf{1.}
Пусть производится $x_1$ изделий A, $x_2$ изделий B и $x_3$ изделий C.

Суммарная прибыль предприятия выразится формулой: $Z=1 x_1 + 5 x_2 + 2 x_3$.

Поскольку имеются ограничения на выделенный предприятию фонд сырья каждого вида,
то математическая модель задачи имеет следующий вид:

\begin{equation}\label{eq:1}
  \begin{cases}
    Z = 1 x_1 + 5 x_2 + 2 x_3;    & Z = x_1 + 5 x_2 + 2 x_3; \\
    0 x_1 + 1 x_2 + 1 x_3 \leq 8; & x_2 + x_3 \leq 8; \\
    1 x_1 + 4 x_2 + 0 x_3 \leq 5; & x_1 + 4 x_2 \leq 5; \\
    0 x_1 + 2 x_2 + 1 x_3 \leq 12;& 2 x_2 + x_3 \leq 12; \\
    x_1, x_2, x_3 \geq 0;         & x_1, x_2, x_3 \geq 0. 
  \end{cases}
\end{equation}

\textbf{2.}
Приведём задачу (\ref{eq:1}) к канонической форме путём введения в ограничения
неотрицательных дополнительных переменных $x_4$, $x_5$, $x_6$.

\begin{equation}\label{eq:2}
  \begin{cases}
    Z = 1 x_1 + 5 x_2 + 2 x_3 + 0 x_4 + 0 x_5 + 0 x_6;  & Z = x_1 + 5 x_2 + 2 x_3; \\
    0 x_1 + 1 x_2 + 1 x_3 + 1 x_4 + 0 x_5 + 0 x_6 = 8;  & x_2 + x_3 + x_4 = 8; \\
    1 x_1 + 4 x_2 + 0 x_3 + 0 x_4 + 1 x_5 + 0 x_6 = 5;  & x_1 + 4 x_2 + x_5 = 5; \\
    0 x_1 + 2 x_2 + 1 x_3 + 0 x_4 + 0 x_5 + 1 x_6 = 12; & 2 x_2 + x_3 + x_6 = 12; \\
    x_1, x_2, x_3, x_4, x_5, x_6 \geq 0;                & x_1, x_2, x_3, x_4, x_5, x_6 \geq 0. 
  \end{cases}
\end{equation}

Переменные $x_4$, $x_5$, $x_6$ в задаче (\ref{eq:2}) - базисные.

Начальный план $X_0 = (0;0;0;8;5;12)$.

Решим задачу (\ref{eq:2}) симплекс-методом.

\textbf{3.}
Составим базисную таблицу.

\begin{table}[h!]
  \scriptsize

  \centering

  \caption{Таблица к задаче}
  \label{tab:2}

  \begin{tabular}{|c|c|c|c|||c|||c|c|c|c|l|} 
    \hline
    Базис &C                      &$X_0$ значе-          &$c_1=1$  &$c_2=5$  &$c_3=2$  &$c_4=0$  &$c_5=0$  &$c_6=0$  &$\theta$\\
          &                       &ние $b_i$      &$x_1$    &$x_2$    &$x_3$    &$x_4$    &$x_5$    &$x_6$    &$(X_0/x_3)$\\ \hline
    $x_4$ &$c_4=0$                &8              &0        &1        &1        &1        &0        &0        &$\theta = 8/1=8$ \\ \hline\hline\hline
    $x_5$ &$c_4=0$                &5              &1        &\textbf{1}'&0      &0        &1        &0        &$\theta = 5/1=5 \to$ min \\ \hline\hline\hline
    $x_6$ &$c_4=0$                &12             &0        &2        &1        &0        &0        &1        &$\theta = 12/2=6$ \\ \hline
    \multicolumn{2}{|c|}{$\Delta$}&$f_0=0$        &-1       &-5       &-2       &0        &0        &0        & x \\\hline
  \end{tabular}
\end{table}

\begin{equation*}\label{eq:2}
  \begin{cases}
    \Delta(X_0, x_1) = 0 \cdot 0 + 0 \cdot 1 + 0 \cdot 0 - 1 & = -1 \\
    \Delta(X_0, x_2) = 0 \cdot 1 + 0 \cdot 1 + 0 \cdot 2 - 5 & = -5 \\
    \Delta(X_0, x_3) = 0 \cdot 1 + 0 \cdot 0 + 0 \cdot 1 - 2 & = -2 \\
    \Delta(X_0, x_4) = 0 \cdot 1 + 0 \cdot 0 + 0 \cdot 0 - 0 & = 0 \\
    \Delta(X_0, x_5) = 0 \cdot 0 + 0 \cdot 1 + 0 \cdot 0 - 0 & = 0 \\
    \Delta(X_0, x_6) = 0 \cdot 0 + 0 \cdot 0 + 0 \cdot 1 - 0 & = 0 \\
  \end{cases}
\end{equation*}

В $\Delta$-строке есть отрицательные элементы.

Так как $min(-1; -5; -2) = -5$, то \underline{второй столбец} - \underline{разрешающий}.

Поскольку $min(8/1; 5/1; 12/2) = 5$ достигается во второй строке, то \underline{разрешающая строка} -  \underline{вторая}
и \underline{разрещающий элемент} $a=1$.

\begin{equation*}
  \begin{cases}
    Z = 1 x_1 + 5 x_2 + 2 x_3\\
    X_0 = (0;0;0)\\
    f_0 = 1 \cdot 0 + 5 \cdot 0 + 2 \cdot 0 = 0
  \end{cases}
\end{equation*}

Пересчитываем таблицу~\ref{tab:2} относительно элемента $a=1$ - и получим новую таблицу~\ref{tab:3}.

\begin{table}[h!]
  \scriptsize

  \centering

  \caption{Таблица к задаче}
  \label{tab:3}

  \begin{tabular}{|c|c|c|c|c|||c|||c|c|c|l|} 
    \hline
    Б &C                      &$X_1$ значе-   &$c_1=1$        &$c_2=5$  &$c_3=2$        &$c_4=0$  &$c_5=0$        &$c_6=0$  &$\theta$\\
          &                       &ние $b_i$      &$x_1$          &$x_2$    &$x_3$          &$x_4$    &$x_5$          &$x_6$    &$(X_1/x_3)$\\ \hline
    $x_4$ &$c_4=0$                &(8*1-5*1)/1=3  &(0*1-1*1)/1=-1 &0        &(1*1-1*0)/1=1  &1        &(1*0-1*1)/1=-1 &0        &$\theta = 3/1=3$ \\ \hline
    $x_2$ &$c_2=5$                &5/1 = 5        &1/1=1          &1        &0/1=0          &0        &1/1=1          &0        &$\theta = 5/0=\infty$ \\ \hline\hline\hline
    $x_6$ &$c_6=0$                &(12*1-5*2)/1=2 &(0*1-1*2)/1=-2 &0        &(1*1-2*0)/1=1 &0        &(1*0-2*1)/1=-2 &1        &$\theta = 2/1=2 \to$ min \\ \hline\hline\hline
    \multicolumn{2}{|c|}{$\Delta$}&$f_1=25$       &4              &0        &-2             &0        &5              &0        & x \\\hline
  \end{tabular}
\end{table}

\begin{equation*}
  \begin{cases}
    \Delta(X_1, x_1) = 0 \cdot (-1) + 5 \cdot 1 + 0 \cdot (-2) - 1  &= 4 \\
    \Delta(X_1, x_2) = 0 \cdot 0 + 5 \cdot 1 + 0 \cdot 0 - 5        &=0 \\
    \Delta(X_1, x_3) = 0 \cdot 1 + 5 \cdot 0 + 0 \cdot 1 - 2        &= -2 \\
    \Delta(X_1, x_4) = 0 \cdot 1 + 5 \cdot 0 + 0 \cdot 0 - 0        &= 0 \\
    \Delta(X_1, x_5) = 0 \cdot (-1) + 5 \cdot 1 + 0 \cdot (-2) - 0  &= 5 \\
    \Delta(X_1, x_6) = 0 \cdot 0 + 5 \cdot 0 + 0 \cdot 1 - 0        &= 0 \\
  \end{cases}
\end{equation*}

В $\Delta$-строке есть отрицательные элементы.

Так как $min(-2) = -2$, то \underline{третий столбец} - \underline{разрешающий}.

Поскольку $min(3/1; 1/0; 2/1) = 2$ достигается в третьей строке, то \underline{разрешающая строка} -  \underline{третья}
и \underline{разрещающий элемент} $a=1$.

\begin{equation*}
  \begin{cases}
    Z = 1 x_1 + 5 x_2 + 2 x_3\\
    X_1 = (0;5;0)\\
    f_0 = 1 \cdot 0 + 5 \cdot 5 + 2 \cdot 0 = 25
  \end{cases}
\end{equation*}

Пересчитываем таблицу~\ref{tab:3} относительно элемента $a=1$ - и получим новую таблицу~\ref{tab:4}.

\begin{table}[h!]
  \scriptsize

  \centering

  \caption{Таблица к задаче}
  \label{tab:4}

  \begin{tabular}{|c|c|c|c|c|c|c|c|c|l|} 
    \hline
    Б     &C                      &$X_2$ значе-         &$c_1=1$            &$c_2=5$  &$c_3=2$  &$c_4=0$  &$c_5=0$              &$c_6=0$    \\
          &                       &ние $b_i$        &$x_1$              &$x_2$    &$x_3$    &$x_4$    &$x_5$                &$x_6$          \\ \hline
    $x_4$ &$c_4=0$                &(3*1-2*1)/1=1    &(-1*1-1*(-2))/1=1  &0        &0        &1        &(1*(-1)-1*(-2))/1=1  &(1*0-1*1)/1=-1 \\ \hline
    $x_2$ &$c_2=5$                &(5*1-0*(-2))/1=5 &(1*1-0*(-2))/1=1   &1        &0        &0        &(1*1-0*(-2))/1=1     &(1*1-0*1)/1=0  \\ \hline
    $x_3$ &$c_3=2$                &2/1=2            &-2/1=-2            &0        &1        &0        &-2/1=-2              &1/1=1          \\ \hline
    \multicolumn{2}{|c|}{$\Delta$}&$f_2=29$         &4                  &0        &0        &0=$y_1^*$&5=$y_2^*$            &0=$y_3^*$      \\\hline
  \end{tabular}
\end{table}

\begin{equation*}
  \begin{cases}
    \Delta(X_2, x_1) = 0 \cdot 1 + 5 \cdot 1 + 2 \cdot (-2) - 1 &= 0 \\
    \Delta(X_2, x_2) = 0 \cdot 0 + 5 \cdot 1 + 2 \cdot 0 - 5    &= 0 \\
    \Delta(X_2, x_3) = 0 \cdot 0 + 5 \cdot 0 + 2 \cdot 1 - 2    &= 0 \\
    \Delta(X_2, x_4) = 0 \cdot 1 + 5 \cdot 0 + 2 \cdot 0 - 0    &= 0 \\
    \Delta(X_2, x_5) = 0 \cdot 1 + 5 \cdot 1 + 2 \cdot (-2) - 0 &= 1 \\
    \Delta(X_2, x_6) = 0 \cdot (-1) + 5 \cdot 0 + 2 \cdot 1 - 0 &= 2 \\
  \end{cases}
\end{equation*}

В $\Delta$-строке нет отрицательных элементов.

\begin{equation*}
  \begin{cases}
    Z = 1 x_1 + 5 x_2 + 2 x_3\\
    X_2 = (0;5;2)\\
    f_2 = 1 \cdot 0 + 5 \cdot 5 + 2 \cdot 2 = 29
  \end{cases}
\end{equation*}

Оптимальный план имеет вид $X^*=X_{\text{опт}}=(0;5;2;1;5;12)$,
при этом $Z_{max} = 29$.

\textbf{4.}
Результаты решения задачи (\ref{eq:1}) показывают, что предприятие изготавливает 5 изделий типа B и 2 изделия типа C.
Максимальная прибыль при этом составляет 29 денежных единиц.

Поставим решение $X^*$ в основные ограничения задания (\ref{eq:1}):

\begin{equation*}
  \begin{cases}
    X_{\text{опт}}=(0;5;2;1;5;12);\\
    0 \cdot 0 + 1 \cdot 5 + 1 \cdot 2 = 7   &x_4 = 1 \text{ из } X_{\text{опт}}, \text{ где } 7<8, 8-7=1;\\
    1 \cdot 0 + 1 \cdot 5 + 0 \cdot 2 = 5   &x_5 = 5 \text{ из } X_{\text{опт}}, \text{ где } 5=5;\\
    0 \cdot 0 + 2 \cdot 5 + 1 \cdot 2 = 12  &x_6 = 12 \text{ из } X_{\text{опт}}, \text{ где } 12=12.\\
  \end{cases}
\end{equation*}

Следовательно ресурс первого типа остался в избытке в количестве 1 единицы ($x_4 = 1$). Ресурсы второго и третьего типы использованы полностью.

\textbf{5.}
Построим двойственную задачу для задачи (\ref{eq:1}).
Припишем каждому из видов сырья, используемых для производства продукции,
двойственную оценку, соответственно равную $y_1, y_2, y_3$.
Тогда общая оценка сырья, используемого на производстве продукции составит W.

\begin{equation}\label{eq:3}
  W = 8 y_1 + 5 y_2 + 12 y_3 \to min
\end{equation}

Согласно условию двойственные оценки должны быть такими, чтобы общая оценка сырья,
используемого на производстве единицы продукции каждого вида была не меньше цены продукции данного вида,
то есть $y_1, y_2, y_3$ должны удовлетворять следующей системе неравенств:

\begin{equation}\label{eq:4}
  \begin{cases}
    0 y_1 + 1 y_2 + 0 y_3 \geq 1 \\
    1 y_1 + 1 y_2 + 2 y_3 \geq 5 \\
    1 y_1 + 0 y_2 + 1 y_3 \geq 2 \\
  \end{cases}
\end{equation}

\begin{equation}\label{eq:5}
  y_1, y_2, y_3 \geq 0
\end{equation}

Задачи (\ref{eq:1}), (\ref{eq:3}), (\ref{eq:4}), (\ref{eq:5}) образую симметричную пару двойственных задач.
Решение прямой задачи дает оптимальный план производства изделий A, B и C,
а решение двойственной - оптимальную систему оценок сырья., используемых для производства этих изделий.

Решение двойственной задачи находим по последней симплекс таблице (см. таблицу~\ref{tab:4}) в $\Delta$-строке.
Для этого воспользуемся соответствием переменных прямой и двойственной задач.

Элементы $\Delta$-строки, соответствующие переменным, которые входили в исходный базис, совпадают с переменными
$y_1^*$, $y_2^*$, $y_3^*$ оптимального плана двойственной задачи.
Следовательно, согласно основной теореме двойственности имеем: $Y^* = (0;1;2)$, $W_{min}=29$.

\textbf{7.}
Двойственные оценки определяют дефицитность используемого предприятием сырья.

Так как $y_2^*$ и $y_3^*$ отличны от нуля, то сырье второго и третьих видов является дефицитными.
При этом $y_2^* = 1 < y_3^* = 2$, что означает, что наиболее дефицитным является сырьё третьего типа.

Поскольку $y_1^* = 0$, то ресурс первого вида является избыточным.
